\documentclass[../../expose]{subfiles}
\usepackage{array}

%Dient nur als Info, es wird das preamble aus dem main-document genommen!
% left fixed width:
\newcolumntype{L}[1]{>{\raggedright\arraybackslash}p{#1}}
% center fixed width:
\newcolumntype{C}[1]{>{\centering\arraybackslash}p{#1}}
% flush right fixed width:
\newcolumntype{R}[1]{>{\raggedleft\arraybackslash}p{#1}}

\begin{document}
Bachelor (Beginn als Beispiel ab Montag den 26.04.2021)
\begin{table}[h]
    \centering
    \begin{tabular}{|C{1cm}|L{8cm}|C{2cm}|C{2cm}| }
        \hline
        Nr.	&Arbeitsschritt&Arbeitstage&Aufwand\\
        \hline
        \hline
        1		&Arbeitsplanung													&X			&1\%\\
        \hline
        2		&Literaturrecherche und-beschaffung					&X			&7\%\\
        \hline
        3		&Literaturauswahl und -auswertung					&X			&13\%\\
        \hline
        X		&Style Guides, OpenAPI, etc. auswerten				&X			&8\%\\
		\hline
		 4		&Strukturieren des Materials								&X			&8\%\\
		\hline
		5		&Erster Gliederungsentwurf									&X			&4\%\\
		\hline
		6      &Gliederungsbesprechungen								&X			&1\%\\
		\hline
		7		&Gliederungsüberarbeitungen								&X			&3\%\\
		\hline
		8		&Manuskripterstellung											&X			&32\%\\
		\hline
		9		&Manuskriptüberarbeitungen								&X			&15\%\\
		\hline
		10	&Endkorrektur, Layoutgestaltung						&X			&3\%\\
		\hline
		11	&Drucken, Binden													&X			&1\%\\
		\hline
		12	&Abgabe																	&X			&1\%\\
		\hline
        13	&Puffer																	&X			&3\%\\
        \hline
        \multicolumn{2}{|R{9cm}|}{Gesamt}&63&100\%\\
        \hline
    \end{tabular}
\end{table}%

\end{document}