\documentclass[../../expose]{subfiles}

\begin{document}

Als Strategie für die Literaturrecherche habe ich vor die systematische Suche anzuwenden, bei der zu Beginn relevante Journals untersucht und darauf basierend ausgewählte Monografien und andere Werke mit einbezogen werden.\newline
%
Das grundsätzliche methodische Vorgehen der Bachelorarbeit ist vermutlich eine Mischung aus Literaturarbeit und qualitativer Inhaltsanalyse. Entsprechend des Vorgehens bei Literaturarbeiten soll relevante Literatur kritisch analysiert, Lücken im Forschungsstand entdeckt , sowie ein Beitrag zum aktuellen Wissenstand des Forschungsgebiets beigesteuert werden. Dieser Beitrag soll in Form von Vorschlägen für Metriken, die eine Aussage über den Qualitätszustand von REST-APIs treffen können sowie die Bewertung dieser Metriken erfolgen. Das Prinzip der qualitativen Inhaltsanalyse wird verfolgt, indem einzelne Style Guides verglichen und analysiert werden um neue Erkenntnisse zu gewinnen. Dabei steht der qualitative Aspekt im Vordergrund, weil es nicht darum geht viele Style Guides zu untersuchen und z.B. Unterschiede aufzuzeigen, sondern einzelne Aspekte grundlegend zu durchdringen und entsprechend zu erörtern. Es handelt in sich diesem Fall allerdings nicht um eine typische Datenerhebung.\newline
Die Argumentation in der Bachelor Thesis erfolgt hauptsächlich induktiv, da vorhandene Daten geprüft und daraus eigene Theorien (Metriken) abgeleitet werden sollen. Ziel ist es neue Erkenntnisse zu gewinnen.

Die neuen Erkenntnisse sollen anhand einer formativen Evaluierung bewertet werden (Analyse-Pespektive ex-ante). Dabei sollen die Metriken qualitativ erforscht werden um einzelne Metriken tiefgehend zu evaluieren.
\end{document}