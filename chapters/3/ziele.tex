\documentclass[../../expose]{subfiles}

\begin{document}
Ziele der Thesis:	
Entwicklung von Anforderungen an qualitativ hochwertig entwickelte APIs. Gewinnung von Metriken die den Qualitätszustand einer REST-API beschreiben und aus den Anforderungen abgeleitet werden. Sinnvolle Kategorisierung der Anforderung sowie Metriken innerhalb eines Katalogs, der zur Qualitätsprüfung von APIs geeignet ist. Es soll eine Grundlage ergründet werden, die darstellt welche Anforderungen und Metriken genutzt werden können, um die Qualität von REST-APIs zu sichern und die durch weitere wissenschaftliche Arbeit erweitert werden kann.	

\textbf{Ziele der Thesis:}
Auswahl von Design Guidelines die für die Aufgabe sinnvoll sind und Erläuterung dieser Entscheidung.
Auswahl von passender wissenschaftlicher Lektüre um das Thema tiefgründig zu durchleuchten, entsprechend zu beschreiben und zu bewerten.
Kritische und qualitätsorientierte Betrachtung der Informationen und Lektüre um Metriken zu entwickeln.
Vorschläge für Metriken die eine Aussage über die Qualität von REST-APIs machen können.
%Beschreibung von Auswirkung, Automatisierungsmöglichkeit, 


Ziele des Projekts:	
Auf Basis des gewonnenen Kataloges der Bachelor Thesis soll ein Analysewerkzeug realisiert werden, dass Software-Entwickler bereits während der Kreation von REST-APIs unterstützt. Es soll visualisiert werden können, welche Entwicklungsstandards fehlerhaft oder noch nicht umgesetzt wurden sowie welche Ergebnisse bei der Prüfung von Metriken erzielt wurden. 	
Nach Anwendung des Analysewerkzeugs auf mindestens eine produktive REST-API sollen die Ergebnisse analysiert und bewertet werden. 	
Weiter soll eine Bewertung der gewählten Art des Analysewerkzeugs erfolgen und inwiefern dieses die Ziele von API-Governance unterstützt.	
Im Rahmen der Case Study soll erkennbar werden, ob das Vorgehen eine solide Grundlage bietet, auf welcher automatisierte Style Guides im Rahmen von API-Governance verwirklicht werden können, um Entwickler zu unterstützen, API-Governance konform zu entwickeln. Dazu soll eine beschreibende Bewertung erstellt werden, die den Sachverhalt schildert.

\textbf{Ziele der Thesis:}

\end{document}