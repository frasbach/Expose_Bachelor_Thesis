\documentclass[../../expose]{subfiles}

\begin{document}

\textbf{Ziele der Thesis:}
\begin{itemize}
\item Auswahl von Design Guidelines die für die Aufgabe sinnvoll sind und Erläuterung dieser Entscheidung.
\item Auswahl von passender wissenschaftlicher Lektüre um das Thema tiefgründig zu durchleuchten, entsprechend zu beschreiben und zu bewerten.
\item Kritische und qualitätsorientierte Betrachtung der Informationen und Lektüre um Metriken zu entwickeln.
\item Vorschläge für Metriken die eine Aussage über die Qualität von REST-APIs machen können.
% \item Beschreibung von Auswirkung, Automatisierungsmöglichkeit, 
\item Vorschlag für überprüfbare Metriken abgeben.
\end{itemize} 



\textbf{Ziele des Projekts:}
\begin{itemize}
\item Auswahl eines geeigneten Tools.
\item Auswahl der Metriken die mit diesem Tool überprüfbar scheinen.
\item Implementierung der ausgewählten Metriken in dem Analysewerkzeug.
\item Anwendung des Analysewerkzeugs auf mindestens eine REST-API.
\item Überprüfung der Ergebnisse sowie der Automatisierungsmöglichkeit.
\item Bewertung der Ergebnisse sowie des Vorgehens und des Tools.
\item Bewertung inwieweit die Ergebnisse eine Aussage über die Qualitätssicherung im Rahmen von API-Governance treffen. \newline
\end{itemize}

\end{document}