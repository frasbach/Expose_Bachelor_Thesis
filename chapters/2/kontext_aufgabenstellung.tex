\documentclass[../../expose]{subfiles}

\begin{document}

Das Gebiet der Programmierschnittstellen und insbesondere deren Qualitätssicherung bieten zum jetzigen Zeitpunkt Potential um weiter erforscht und optimiert zu werden. Es gibt bereits öffentliche Guidelines die beschreiben wie APIs entwickelt werden, dokumentierte Best Practices und auch Konzeptvorschläge wie man die Entwicklung mehrerer APIs umsetzten kann. Derzeit fehlt es  aber noch an Möglichkeiten die Qualität von Softwareschnittstellen zu überprüfen. 
Die viadee AG arbeitet an mehreren modernen, verteilten Software Systemen, die über REST-Schnittstellen kommunizieren und ist interessiert die Entwicklung der Qualitätssicherung von APIs zu unterstützen. 

Im Laufe der zweiten Praxisphase sollen reputable und öffentliche Design Guidelines für REST-APIs analysiert  und das daraus gewonnene Wissen durch wissenschaftliche Quellen gestützt werden. Bei der Evaluierung soll der Fokus nicht auf einer API für sich alleine liegen sondern auch die Aspekte des API-Manangements und API-Governance in die Bewertung mit einfließen. Auf Basis der Ergebnisse sollen, wenn möglich Metriken entworfen werden, mit denen eine Aussage über den qualitativen Zustand von APIs getroffen werden kann.

Auf dieser Grundlage sollen anschließend  ausgewählte Metriken in einem Analysewerkzeug implementiert werden um APIs zu prüfen und bewerten zu können. Im Rahmen dieser Case-Study soll das Analysewerkzeug auf APIs angewendet und  bewertet werden. 

Die Bachelorarbeit lässt sich den Modulen Business Engineering und Software Engineering zuordnen. Außerdem lassen sich Teile des Requirements Engineering abgewandelt übernehmen.


% TODO Schaubild einfügen
\end{document}