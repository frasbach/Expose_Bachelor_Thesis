\documentclass[../../expose]{subfiles}

\begin{document}

Die viadee AG arbeitet an mehreren modernen, verteilten Software Systemen, die über REST-Schnittstellen kommunizieren. Im Laufe der zweiten Praxisphase sollen reputable und öffentliche Design Guidelines für APIs miteinander verglichen, sowie mit weiteren wissenschaftlichen Quellen überprüft werden. Das gewonnene Wissen soll in Form von Anforderungen gebündelt und aus den Anforderungen wo möglich Metriken abgeleitet werden. Mithilfe der gesammelten Anforderungen soll überprüft werden können ob eine REST-API konform zu firmeninternen Richtlinien entwickelt wird. Anhand der gewonnen Metriken soll die Qualität von REST-APIs beurteilt werden können.
Anforderungen und Metriken sollen zumindest auf Grund des Zeitpunkts der Überprüfung unterschieden werden, da die Bachelor Thesis als Grundlage dienen soll, um ein Analysewerkzeug zu bauen, das ausgewählte, vorher ergründete Anforderungen und Metriken überprüfen kann, um Entwickler bei der Erstellung von REST-APIs zu unterstützen.
Im zweiten Teil der Praxisphase soll auf Basis der Erkenntnisse ein Analysewerkzeug entwickelt werden, das ausgewählte Anforderungen und Metriken implementiert um APIs auf Konformität zu Richtlinien sowie auf Qualität zu prüfen. Anschließend soll eine sinnvolle Darstellung der Ergebnisse erfolgen die Entwickler bei der Entwicklung von APIs unterstützt.  Das Analysewerkzeug soll danach gegen API-Restschnittstellen getestet und die Ergebnisse bewertet werden. 	
Die Bachelorarbeit lässt sich den Modulen Business Engineering und Software Engineering zuordnen. Teile des Requirements Engineering lassen sich abgewandelt übernehmen. 

\end{document}