\documentclass[12pt,a4paper,oneside,ngerman]{article}

\usepackage{subfiles}
\usepackage[a4paper,left=4.5cm,right=2.5cm,top=2.5cm,bottom=2cm]{geometry}
\usepackage{fancyhdr} % Kopf- und Fußzeile
\usepackage[onehalfspacing]{setspace} % 1,5 facher Zeilenabstand
\usepackage{graphicx} % Bilder einbinden
\usepackage{titling} % Referenzen auf Titel / Autor etc. im Text
\usepackage[hidelinks]{hyperref} % Hyperlinks im Inhaltsverzeichnis
\usepackage[german]{babel} % Übersetzungen
\usepackage{acronym} % Abkürzungsverzeichnis
\usepackage[backend=biber,style=LNI]{biblatex} % Literaturverzeichnis
\usepackage{csquotes} % Babel und Biblatex Sprachanpassungen
\usepackage{framed} % Kästen für den Sperrvermerk
\addbibresource{literature.bib}

\author{Florian Rasbach}
\title{Expose zur Bachelor Thesis}
\newcommand{\degreeCourse}{Wirtschaftsinformatik}
\newcommand{\studentNumber}{967750}
\newcommand{\company}{viadee AG}
\newcommand{\filingDate}{26 Juli 2021-}
\newcommand{\firstExaminer}{Professor Dr. Sebastian Thöne}
\newcommand{\secondExaminer}{-Hier Zweitprüfer eintragen/-}
\providetoggle{blockingNotice}
\settoggle{blockingNotice}{true} % Sperrvermerk anzeigen oder ausblenden
\setlength{\parindent}{0em}

\setlength{\headheight}{14.5pt} % Damit fancyhdr keine Fehlermeldung generiert
\setlength{\emergencystretch}{1.5em} % Für passende Wortumbrüche


% Header und Footer Definitionen
%---------------------------------------------
\fancypagestyle{onlyPageNumbers}{ % Zeigt nur Footer mit Seitenzahl rechts
    \fancyhf{}
    \fancyfoot[C]{\thepage}
    \fancypagestyle{plain}{%
        \fancyhf{}%
        \fancyfoot[R]{\thepage}%
        \renewcommand{\headrulewidth}{0pt}
    }
}

\fancypagestyle{toc}{ % Zeigt Footer + Header auf zweiter Seite (ToC)
    \fancyhf{}
    \fancyfoot[C]{\thepage}
    \fancypagestyle{plain}{%
        \fancyhf{}%
        \fancyhead[R]{\nouppercase{\leftmark}}
        \fancyfoot[R]{\thepage}%
    }
}

\pagestyle{onlyPageNumbers}


\begin{document}

% D E C K B L A T T
%---------------------------------------------
\newgeometry{margin=3cm} % Seitenränder für Deckblatt anders

\begin{titlepage}
    \begin{center}
        \includegraphics[scale=1.25]{res/fh_logo.pdf}
    \end{center}

    \vspace*{\stretch{1.0}}
    \begin{center}
        \Large\textbf\thetitle\\
        \vspace{1em}
        \large{Abschlussarbeit im Studiengang {\degreeCourse}}\\
        {am Fachbereich Wirtschaft der FH Münster}

    \end{center}
    \vspace*{\stretch{2.0}}

    \begin{center}
        \begin{tabular}{l l}
            {Vorgelegt von:} & \theauthor \\
            {Matrikelnummer:} & \studentNumber \\
            {Erstprüfer:} & {\firstExaminer} \\
            {Zweitprüfer:} & {\secondExaminer} \\
            {Unternehmen:} & {\company} \\
            {Abgabedatum:} & {\filingDate} \\
        \end{tabular}
    \end{center}
\end{titlepage}

\restoregeometry % Seitenränder auf Standard setzen

% H A U P T T E I L
%---------------------------------------------
\section {Einführung zu APIs und API-Governance \textsl{(Kann übersprungen werden!)}}
\subfile{chapters/1/einfuehrung}

\section {Kontext und Aufgabenstellung}
\subfile{chapters/2/kontext_aufgabenstellung}

\section {Kontext und Aufgabenstellung}
\subfile{chapters/3/ziele}

\end{document}